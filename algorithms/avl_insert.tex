% avl_insert.tex
% 11-19-2001
\nopagenumbers
\parskip=2pt
\parindent=0pt

{\sl An Iterative Algorithm for AVL Tree Insertion
	\hfill \rm Farooq Mela $<$farooq.mela@gmail.com$>$}

\medskip

\noindent {\bf Algorithm A} ({\it AVL Tree Insertion}).
Given a set of nodes which form an AVL tree {\tt T}, and a key to insert
{\tt K}, this algorithm will insert the node into the tree while maintaining
the tree's balance properties. Each node is assumed to contain {\tt KEY},
{\tt BAL}, {\tt LLINK}, {\tt RLINK}, and {\tt PARENT} fields. For any given
node {\tt N}, {\tt KEY(N)} gives the key field of {\tt N}, {\tt BAL(N)} gives
the balance field of {\tt N}, {\tt LLINK(N)} and {\tt RLINK(N)} are pointers to
{\tt N}'s left and right subtrees, respectively, and {\tt PARENT(N)} is a
pointer to the node of which {\tt N} is a subtree. Any or all of these three
link fields may be $\Lambda$, which for {\tt LLINK(N)} and {\tt RLINK(N)}
indicates that {\tt N} has no left or right subtree, respectively, and for
{\tt PARENT(N)} indicates that {\tt N} is the root of the tree. The {\tt BAL}
field must be able to represent an integer on the range $[-2, +2]$. The tree
has a field {\tt ROOT} which is a pointer to the root node of the tree.

\medskip
You can find an implementation of this algorithm, as well as many others, in
{\bf libdict}, which is available on the web at
{\tt http://www.crazycoder.org/libdict.html}.
\medskip

\parindent=36pt

\item{\bf A1.} [Initialize.]
Set ${\tt N} \gets {\tt ROOT(T)}$, ${\tt P} \gets {\tt Q}\gets \Lambda$.

\item{\bf A2.} [Find insertion point.]
If ${\tt N} = \Lambda$, go to step A3.
If ${\tt K} = {\tt KEY(N)}$,
the key is already in the tree and the algorithm terminates with an error.
Set ${\tt P} \gets {\tt N}$;
if ${\tt BAL(P)} \neq 0$, set ${\tt Q} \gets {\tt P}$.
If ${\tt K} < {\tt KEY(N)}$,
then set ${\tt N} \gets {\tt LLINK(N)}$,
otherwise set ${\tt N} \gets {\tt RLINK(N)}$. Repeat this step.

\item{\bf A3.} [Insert.]
Set ${\tt N} \gets {\tt AVAIL}$.
If ${\tt N} = \Lambda$, the algorithm terminates with an out of memory error.
Set ${\tt KEY(N)} \gets {\tt K}$,
${\tt LLINK(N)} \gets {\tt RLINK(N)} \gets \Lambda$,
${\tt PARENT(N)} \gets {\tt P}$, and
${\tt BAL(N)} \gets 0$.
If ${\tt P} = \Lambda$, set ${\tt ROOT(T)} \gets {\tt N}$, and go to step A8.
If ${\tt K} < {\tt KEY(P)}$,
set ${\tt LLINK(P)} \gets {\tt N}$; otherwise,
set ${\tt RLINK(P)} \gets {\tt N}$.

\item{\bf A4.} [Adjust balance factors.]
If ${\tt P} = {\tt Q}$, go to step A5.
If ${\tt LLINK(P)} = {\tt N}$,
set ${\tt BAL(P)} \gets -1$; otherwise,
set ${\tt BAL(P)} \gets +1$.
Then set ${\tt N} \gets {\tt P}$, and ${\tt P} \gets {\tt PARENT(P)}$,
and repeat this step.

\item{\bf A5.} [Check for imbalance.] If ${\tt Q} = \Lambda$, go to step A8.
Otherwise:

\itemitem{i.} If ${\tt LLINK(Q)} = {\tt N}$,
set ${\tt BAL(Q)} \gets {\tt BAL(Q)} - 1$.
If ${\tt BAL(Q)} = -2$, go to step A6, otherwise, go to step A8.

\itemitem{ii.} If ${\tt RLINK(Q)} = {\tt N}$,
set ${\tt BAL(Q)} \gets {\tt BAL(Q)} + 1$.
If ${\tt BAL(Q)} = +2$, go to step A7, otherwise, go to step A8.

\item{\bf A6.} [Left imbalance.]
If ${\tt BAL(LLINK(Q))} > 0$, rotate {\tt LLINK(Q)} left. Rotate {\tt Q} right.
Go to step A8.

\item{\bf A7.} [Right imbalance.]
If ${\tt BAL(RLINK(Q))} < 0$, rotate {\tt RLINK(Q)} right. Rotate {\tt Q} left.
Go to step A8.

\item{\bf A8.} [All done.] The algorithm terminates successfully.

\medskip
\parindent=0pt
{\bf Rotations in AVL Trees}

\noindent {\bf Algorithm R} ({\it Right Rotation}).
Given a tree {\tt T} and a node in the tree {\tt N}, this routine will rotate
{\tt N} right.

\parindent=36pt
\item{\bf R1.} [Do the rotation.]
Set
${\tt L} \gets {\tt LLINK(N)}$ and
${\tt LLINK(N)} \gets {\tt RLINK(L)}$.
If ${\tt RLINK(L)} \neq \Lambda$,
then set ${\tt PARENT(RLINK(L))} \gets {\tt N}$.
Set ${\tt P} \gets {\tt PARENT(N)}$, ${\tt PARENT(L)} \gets {\tt P}$.
If ${\tt P} = \Lambda$, then set ${\tt ROOT(T)} \gets {\tt L}$;
if ${\tt P} \neq \Lambda$ and ${\tt LLINK(P)} = {\tt N}$,
set ${\tt LLINK(P)} \gets {\tt L}$, otherwise
set ${\tt RLINK(P)} \gets {\tt L}$.
Finally, set ${\tt RLINK(L)} \gets {\tt N}$,
and ${\tt PARENT(N)} \gets {\tt L}$.

\item{\bf R2.} [Recompute balance factors.]
Set
${\tt BAL(N)} \gets {\tt BAL(N)} + (1 - {\tt MIN(BAL(L), 0)})$,
${\tt BAL(L)} \gets {\tt BAL(L)} + (1 + {\tt MAX(BAL(N), 0)})$.

\parindent=0pt

\medskip
The code for a left rotation is symmetric. At the risk of being repetitive, it
appears below.
\medskip

\noindent {\bf Algorithm L} ({\it Left Rotation}).
Given a tree {\tt T} and a node in the tree {\tt N}, this routine will rotate
{\tt N} left.

\parindent=36pt
\item{\bf L1.} [Do the rotation.]
Set
${\tt R} \gets {\tt RLINK(N)}$ and
${\tt RLINK(N)} \gets {\tt LLINK(R)}$.
If ${\tt LLINK(R)} \neq \Lambda$,
then set ${\tt PARENT(LLINK(R))} \gets {\tt N}$.
Set ${\tt P} \gets {\tt PARENT(N)}$, ${\tt PARENT(R)} \gets {\tt P}$.
If ${\tt P} = \Lambda$, then set ${\tt ROOT(T)} \gets {\tt R}$;
if ${\tt P} \neq \Lambda$ and ${\tt LLINK(P)} = {\tt N}$,
set ${\tt LLINK(P)} \gets {\tt R}$, otherwise
set ${\tt RLINK(P)} \gets {\tt R}$.
Finally, set ${\tt LLINK(R)} \gets {\tt N}$,
and ${\tt PARENT(N)} \gets {\tt R}$.

\item{\bf L2.} [Recompute balance factors.]
Set
${\tt BAL(N)} \gets {\tt BAL(N)} - (1 + {\tt MAX(BAL(R), 0)})$,
${\tt BAL(R)} \gets {\tt BAL(R)} - (1 - {\tt MIN(BAL(N), 0)})$.

\parindent=0pt

\bye
% EOF
